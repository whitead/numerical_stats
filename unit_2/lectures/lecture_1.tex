\documentclass{article}

\usepackage{teaching, array, amsmath}

\renewcommand{\Pr}{\textrm{P}}

\begin{document}

\begin{tdoc}{CHEM 116}{Unit 2, Lecture 1}{Numerical Methods and Statistics}

  \subsection*{Companion Reading}
  \textbf{Bulmer} Chapter 3


  \section{Probability II}

  \subsection{Combinations and Permutations}

  We'll take a quick detour and learn about counting, or combinatorial math.

  {\bf If you have $n$ objects and are making samples of length $l$, there are $n^l$ permutations}

  This one is quite simple. Think about a number. You know the number
  of digits places and 0-9 can go in each. So $l$ is the number of
  digits places and $n$ is the number of possible numbers. For example, there
  are $10^4$ possible 4 digit numbers (0000-9999), as expected.

  {\bf  If you have $n$ items and it happens that you're making samples of length $n$, the number of permutations is $n!$. }

  Consider the three letters in the word `cat'. You can rearrange this many ways:

  \begin{enumerate}
  \item cat
  \item cta
  \item tac
  \item tca
  \item atc
  \item act
  \end{enumerate}

  See the pattern of three possibilities for the first letter. Then,
  for each first letter there are two possibilities. Then knowing the
  first two letters, there is only one possibility for the last: $3 \times 2 \times 1$.

  {\bf if you have $n$ items and are making samples of length $l$, but
    you disallow repeats (no replacement) of the $n$ items you have
    this many permutations:}

  $$  \frac{n!}{(n - l)!}$$

  Disallowing repeats here means, like above, if we use a letter once
  we cannot use it again in the word. We can't make `cc' from the
  letters `cat' because the `c' cannot be repeated. This is usually
  called \textbf{without replacement}. Once we use a `c', we do not
  replace it in our list of available items.

  The equation above is similar to the above, but our factorial starts
  at $n$ and counts down until we've exhausted $l$. For example, if we
  have the letters A, B, C, D and E and are making all 3-letter
  combinations, we have: $5 \times 4 \times 3$. That's what the
  equation above is.

  {\bf if you have $n$ items and are making samples of length $l$, but
    you disallow repeats (without replacement) of the $n$ items you have
    this many combinations:}

  $$  \frac{n!}{l!(n - l)!} = \binom{n}{l}$$

  This is the same as above, but divided by $l!$ since permutations of
  the same combination are not counted twice (e.g., 1, 2, 3 is the
  same combination as 3, 2, 1). How did we know there are $l!$
  permutations for each combination? From the rule above.

  {\bf if you have $n$ items and are making samples of length $l$, and
    you allow repeats (with replacement) of the $n$ items you have
    this many combinations:}

  $$
  \frac{(n + l - 1)!}{l!(n - 1)!}
  $$

  This one is a little harder to understand. Look up ``Stars and
  Bars'' on Wikipedia which is a well-known derivation of this formula. {\it One
    important detail is that we can now have $l > n$ or $l < n$ since we allow
    repeats}. Let's see an example. Consider the letters D, F, G and
  I'm making combinations of length 2:

  \begin{enumerate}
  \item DD
  \item FF
  \item GG    
  \item DF
  \item DG
  \item FG
  \end{enumerate}

  Notice we don't have FD in our list because that's the same combination as DF.  Using our equation above we get

  $$
  \frac{(3 + 2 - 1)!}{2!(3 - 1)!} = \frac{4\times3\times2\times1}{2\times1\times2\times1} = \frac{24}{4} = 6
  $$

  For comparison, the number of permutations here is $3^2 = 9$. The
  three items that that are permutations but not combinations are FD,
  DG, GF.

  You now know all the possible cases for combinations and
  permutations! Good job!

  \subsubsection{More Examples}

  Let's say my sample space is 5 digit numbers. How large is it?\\
  $$10^5$$

  Given that there are four possible pets (cats, foxes, dogs, horse)
  and everyone owns 2, what's the sample space size of pet
  ownership?\\ If we disallow multiple pets of the same type:
  $\frac{4!}{(4 - 2)!} = 12$. Or $4^2 = 16$

  What's the sample space size of a shuffled deck of cards?\\
  $$52! \approx 10^{68}$$
  for reference there are $10^{57}$ atoms in the solar system

  What's the sample space size for a draw of 5 cards?\\
  $$\frac{52!}{5!(52 - 5)!} = 2,598,960$$

  How many combinations of rolls of 2 dices are possible?\\
  $$\frac{(6 + 2 - 1)!}{2!(6 - 1)!} = 21$$

\subsection{Repeated Items in Permutations}

What happens when one of our $n$ items has a repeat? For example, consider rearranging
the letters in the word `DAD':

\begin{enumerate}
\item DAD
\item DDA
\item ADD
\end{enumerate}

However, our previous rule tells us that there should be $3! = 6$ permutations. The reason why we don't have 6 is that we `lose' some permutations because
the two $D$s are interchangeble. For each of the words we could swap the two Ds and keep the same word. Thus for each two permutations according
to our original rule, there is actually only one unique permutations:  $$ \frac{3!}{2} = 3$$

Let's now look at three repeated items: `ABBB':
\begin{enumerate}
\item ABBB
\item BABB
\item BBAB
\item BBBA
\end{enumerate}

We see 4 permutations and our rule says $4!=24$. To see what's happening, consider one of our unique permutations but label the letters:
\begin{itemize}
\item AB$_1$B$_2$B$_3$
\item AB$_1$B$_2$B$_3$
\item AB$_3$B$_2$B$_1$
\item ...
\end{itemize}
Thus for each of our unique permutations, there are $3!$ ways to rearrange but keep the same permutation. Put another way, of the 24 rearrangements,
we over-count by $3!$.

Following similar logic you can see that if we have multiple repeated items, the product of their repeats shows up in the denominator. For example, if you have `ABBBDDCE', the number of permutations is:
\[
\frac{8!}{3!\times2!} = 3360  
\]

Symbolically this is:
\begin{equation}
  \frac{n!}{\prod_i k_i!}
\end{equation}

\subsection{Multiple Combinations}
Consider if we don't specify $l$ for combinations. For example, say you have $n$ items and you can make all combinations from $l=0$ through $l=n$. To analyze this, we can just sum our rules for each $l$:

\[
  \sum_{l=0}^n \frac{n!}{l!(n - l)!}  
\]
This expression can be simplified:
\begin{equation}
  \sum_{l=0}^n \frac{n!}{l!(n - l)!} = 2^n
\end{equation}

This can be thought of as the number of subsets you can make. Subsets, like we saw for events, are sets of sets. For example if you have a set consisting of $\{3, 5, 2\}$, the possible subsets are:
\begin{enumerate}
\item $l=0$, $\{\}$
\item $l=1$, $\{3\}$
\item $l=1$, $\{5\}$
\item $l=1$, $\{2\}$
\item $l=2$, $\{2, 3\}$
\item $l=2$, $\{2,5\}$
\item $l=2$, $\{3,5\}$
\item $l=3$, $\{2, 3, 5\}$
\end{enumerate}

So there are $2^3 = 8$ combinations, including the `empty' combination. Notice that when $n = l$, there is one combination. Don't get confused: when $n = l$ there is 1 combination and $n!$ permutations.

\subsection{Multidimensional Sample Spaces}

Nearly all examples in class thus far used sample spaces that have a single
dimension. Sample spaces can be pairs of values in a 2D space or
higher.

\begin{itemize}
\item the sample space could be the roll of two dice
\item The sample space could be GPS coordinates
\end{itemize}

Sample spaces can be joined together into {\bf product spaces},
indicated as $Q_1 \otimes Q_2$. A product space is every possible
pairing of two spaces, meaning the number of elements is the product
of the number of elements of the component spaces ($|Q| = |Q_1| \times
|Q_2|$).

\begin{itemize}

\item Color of my shirt by weather today
\item 5 tempertaure sensors
\item Concentration of reactant A, B and product C
\item The roll of two die is a product space, $6\times 6$
\item The roll of $n$-die is $6^n$
\item $\{-1,0,1\} \otimes \{0,1\} = \{(-1,0), (0,0), (1,0), (-1,1), (0,1), (1,1)\}$

\end{itemize}

Note we are not observing two samples now where we previously observed
one. Instead we are observing samples composed of multiple values
(tuples).

\subsection{Event vs Sample on Product Spaces}
Is rolling a $5$ a sample or event? What if our space is rolling two
dice? Either $n=1,\; Q=\{1,\ldots,6\}$ or $n=2\times 6 - 1,\; Q=\{1,\ldots,6\} \otimes \{1,\ldots,6\}$.
$1/6$ vs $11/36$.

What about weather by day of the week. If it's snowing is now an
event. Why?
\[
\Pr(\textrm{snowing}) = \Pr(\textrm{snowing and Monday}) + \Pr(\textrm{snowing and Tuesday}) + \ldots
\]

\section{Random Variables}

A random variable is a \emph{function} whose domain is a sample space
and whose range is continuous or discrete values. It is like a
generalization of events. Events are true or false, but random
variables take on any possible numerical value. Because a random
variable is defined in a sample space, we may compute probabilities
for each of its possible outputs.

\subsection{Examples of random variables}

\begin{itemize}

\item On the sample space of the roll of a die, the value of the
  sample (1,2,3,4,5,6) is an rv.


\item On the sample space of the roll of a die, the square of the
  sample (1,4,9,16,25,36) is an rv.

\item On the pathway example from homework, the
  length of a path is the rv.

\item On the sample space of temperature, the value of
  temperature is an rv.

\item On the sample space of temperature, an indicator of 0,1 if the
  temperature is above a set value is an rv.

\end{itemize}

\subsection{Writing random variables}

Random variables are written like $X$ and the probability that an rv
takes a specific value, $x$ is written as $\Pr(X=x)$. This is called a
{\bf probability mass function}. The reason we introduce $x$ is that
the right-hand-side typically depends on $x$. For example, consider
our biased die from last lecture where the relative probabilities of a
roll follow the value of the roll (e.g., a 2 is twice as likely as a
1). If the rv is the value of the roll:
\[
\Pr(X=x) = \frac{x}{21}
\]
if the rv is the square:
\[
\Pr(Y=y) = \frac{\sqrt{y}}{21}
\]
\[
\Pr(Y=25) = \frac{\sqrt{25}}{21} = \frac{5}{21}
\]

\emph{Almost always, we will omit the $X$ in the probability mass
  function and only write $\Pr(x)$ instead of $\Pr(X=x)$}

\subsection{Random Variables in Multidimensional Sample Spaces}
This adds complexity in the rv on the sample space. Let's take the
sample space of lattitude and longitude. We could define two rvs, $X$
and $Y$ that are the lattitude and longitude, respectively. However,
we could also define $Z$ which is the product of lattitude and
longitude.

{\bf Do not confuse the dimension of the sample space and the number of rvs. They are not the same.}

\subsection{Complexity of Continuous Random Variables}
We are measuring the concentration of a reaction, $\theta$ and the
sample space is 0 to 20 M. We note a value, $2.55$. What is
$P(\Theta=2.55)?$ Discuss

We will never ever use probability mass functions with continuous
random variables. It makes no sense. Instead, we always deal with
intervals. For example, what is the probability of the concentration
being between $2.549$ and $2.501$? That can be computed using the {\bf
  probability density function} (PDF). The PDF may be thought of as
the pseudo-derivative of a probability mass function. It's used like so:
\[
\Pr(2.549 < x < 2.501) = \int_{2.549}^{2.501} p(x)\,dx
\]
Notice an integral is used to treat intervals and the PDF ($p(x)$) is
indicated with a lower-case.

{\bf We will never deal with single points in continuous rv and you
  should always be thinking in terms of intervals over the PDF
  instead}.

Example: Uniform distribution
\[
p(x) = 1
\]
due to normalization
\[
\int_L^U p(x) \,dx = 1,\:\Rightarrow p(x) = \frac{1}{U-L}
\]


\end{tdoc}

\end{document}
